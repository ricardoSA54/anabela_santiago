— Cri! Cri! Cri! Ouviste como são barulhentos os humanos? – diz um grilo que emudece sempre que ouve vozes de gente. Mas desta vez não se pôs em fuga. Noutra ocasião teria fugido a sete patas.
\bigbreak
\textbf{Eu sou Eu} levanta-se da relva, sacode as pequenas folhas que caíram no seu traje prateado e diz:
\bigbreak
— É verdade que os humanos têm formas estranhas de se comunicar. Umas vezes falam, outras gritam e noutras sussurram. É conforme as emoções que vão sentindo. Riem, choram, ficam sérios e por vezes praguejam. Nessas alturas nem eles se entendem, porque raramente se silenciam, permitindo-se ouvir o que lhes é dito. Atropelam-se com os próprios pensamentos desnorteados e atropelam os outros com palavras. Usam-nas como armas. A Joana não ouviu o primeiro chamamento da mãe, porque estava concentrada na sua brincadeira. Tudo à sua volta deixou de existir enquanto cantava e sentia a relva nos pés, o vento na cara, o calor do sol no corpo. Vivia o momento Presente, o Aqui e o Agora. As crianças não se preocupam com o Passado e nem com o Futuro. Simplesmente saboreiam o Presente, como se fosse um brinquedo novo que acabaram de receber. Por essa razão se chama “presente”. E por essa mesma razão estão sempre felizes.
\bigbreak
Os insetos ouvem com atenção a voz suave do \textbf{Eu sou Eu} colocando o seu ponto de vista relativamente aos humanos. Ele continua:
\bigbreak
— Assim, aqui na Terra, por vezes dizem que as crianças estão no Mundo da Lua. E é completamente verdade. Estão na lua ou ainda mais longe. Elas têm a capacidade de fazer viagens para outros planetas sem saírem aqui da Terra. Em AnThais, contatamos com muitas que nos visitam. É um prazer para nós ouvirmos as suas histórias e aventuras. E existe coisa melhor do que viajar para onde se quer? Olhem para mim, eu sou de um outro Mundo!
\bigbreak
\textbf{Eu sou Eu} solta uma gargalhada e contagia todos com o seu riso. É de tal maneira intenso que as quarenta e oito árvores sentinelas ajustam a postura dos seus troncos, levando, inclusive, a calar os pássaros tagarelas que também tentam perceber o que está a acontecer.

Os insetos, sem controlarem o riso, começam a rebolar com as patas na barriga e alguns chegam mesmo a chorar de alegria.

Sem dúvida, a Primavera tem uma energia estranha que em tudo se entranha. Em tudo o que está em harmonia com ela.
\bigbreak
E neste dia, em que começa a Primavera, os insetos já sentem o Visitante de um outro Mundo, aterrissado no seu jardim há tão pouco tempo, como fazendo parte da sua vida desde sempre.

Neste Mundo, não há acasos, tudo tem um propósito.
